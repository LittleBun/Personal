%
% This is a borrowed LaTeX template file for lecture notes for CS267,
% Applications of Parallel Computing, UCBerkeley EECS Department.
% Edited and being used by Yao Zheng. 
%

\documentclass{article}
\setlength{\oddsidemargin}{0.25 in}
\setlength{\evensidemargin}{-0.25 in}
\setlength{\topmargin}{-0.6 in}
\setlength{\textwidth}{6.5 in}
\setlength{\textheight}{8.5 in}
\setlength{\headsep}{0.75 in}
\setlength{\parindent}{0 in}
\setlength{\parskip}{0.1 in}

%
% ADD PACKAGES here:
%

\usepackage{listings}

\lstdefinelanguage
   [x64]{Assembler}     % add a "x64" dialect of Assembler
   [x86masm]{Assembler} % based on the "x86masm" dialect
   % with these extra keywords:
   {morekeywords={LW,ADDIU,CDQE,CQO,CMPSQ,CMPXCHG16B,JRCXZ,LODSQ,MOVSXD, %
                  POPFQ,PUSHFQ,SCASQ,STOSQ,IRETQ,RDTSCP,SWAPGS}} % etc.

\lstset{language=[x64]Assembler}

\usepackage{amsmath,amsfonts,graphicx,hyperref}
\usepackage[backend=biber,style=alphabetic]{biblatex}
\addbibresource{./ee660_2017_spring_ref.bib}

%
% The following commands set up the assgnum (assignment number)
% counter and make various numbering schemes work relative
% to the assignment number.
%
\newcounter{assgnum}
\renewcommand{\thepage}{\theassgnum.\arabic{page}}
\renewcommand{\thesection}{\theassgnum.\arabic{section}}
\renewcommand{\theequation}{\theassgnum.\arabic{equation}}
\renewcommand{\thefigure}{\theassgnum.\arabic{figure}}
\renewcommand{\thetable}{\theassgnum.\arabic{table}}

%
% Use this macro to generate the header.
% usage: \assignment{COURSE NUMBER}{COURSE NAME}{SEMESTER}{ASSIGNMENTS NUMBER}{DUE DATE}{TOTAL SCORE}{STUDENT NAME}{LAST MODIFIED DATE}
\newcommand{\assignment}[8]{
   \pagestyle{myheadings}
   \thispagestyle{plain}
   \newpage
   \setcounter{assgnum}{#4}
   \setcounter{page}{1}
   \noindent
   \begin{center}
   \framebox{
      \vbox{\vspace{2mm}
      \hbox to 6.28in { {\bf #1: #2 \hfill #3} }
      \vspace{4mm}
      \hbox to 6.28in { {\Large \hfill Assignment No. #4 (Due: #5)  \hfill} }
      \vspace{4mm}
      \hbox to 6.28in { {\Large \hfill Score: $\rule{1cm}{0.1mm}$/#6  \hfill}}
      \vspace{2mm}
      \hbox to 6.28in { {\it Student: #7 \hfill Date: #8} }
      \vspace{2mm}}
   }
   \end{center}
   \markboth{Assignment No. #4 (Due: #5)}{Assignment No. #4 (Due: #5)}

}

%Use this command for a figure; it puts a figure in wherever you want it.
%usage: \fig{NUMBER}{SPACE IN INCHES}{CAPTION}{PATH-TO-FIGURE}
\newcommand{\fig}[4]{
         \begin{center}
         \includegraphics[height=#2]{#4}\\
         Figure \theassgnum.#1:~#3
         \end{center}
   }

%
% The following commands set up the probnum and solnum for problems and solutions
%
\newcounter{probnum}
\newcounter{solnum}

% Use this command for problem
% usage: %\problem{SCORE}{TOTAL SCORE}
\newcommand{\problem}[2]{%
  \stepcounter{probnum}%
  \textbf{\theassgnum.\theprobnum~(#1/#2).}~%
  \setcounter{solnum}{0}%
}

% Use this command for solutions
% usage: %\solution{SUB PROBLEM NUM}
\newcommand{\solution}{%
  \stepcounter{solnum}%
  \textbf{Solution~\theassgnum.\theprobnum.\thesolnum:}\\%
}

%  IF YOU WANT TO DEFINE ADDITIONAL MACROS FOR YOURSELF, PUT THEM HERE:

\begin{document}
%FILL IN THE RIGHT INFO.
\newcommand{\totalscore}{100}
\assignment{EE660}{Computer Architecture}{Spring, 2017}{2}{Feburary 24, 2017}{\totalscore}{your name, your email}{\today}

\problem{20}{100} Draw the pipeline diagram of the following code executing on the shown in-order two-way superscalar processor. Assume that branches execute in pipeline A, loads/stores in pipeline B. Assume full bypassing when possible and no alignment problems.
\fig{1}{2in}{Two-Wide In-order Superscalar Processor Pipeline.}{2_wide_superscalar.png}
\begin{lstlisting}
ADD R5, R6, R7
SUB R6, R7, R8  
LW R10, R6(0) 
ADDIU R12, R13, 1 
LW  R15, R6(4)
LW  R15, R15(4) 
ADD R6, R9, R10 
ADDIU R8, R10, R11
\end{lstlisting}

% \solution Your solution ...

\problem{20}{100} Given that you have an architecture which has the following pipeline stages:
\begin{lstlisting}
F D I X0 X1 W
\end{lstlisting}
and that register fetch happens in the I stage of the pipeline and branch resolution happens in X1, how many dead instructions are need to be killed when a branch miss-predict is taken? Now assume that the pipeline is a three-wide superscalar, how many instructions need to be killed on a branch miss-predict? (Assume that the branch is the first instruction executing after a jump.)


% \solution Your solution ...
\newpage

\problem{20}{100} \cite{PaHe:11}, page 248-249, problem 3.3-3.4.

% \solution Your solution...

\problem{20}{100} Assume that you have the IO2I pipeline from lecture. It can issue one instruction per cycle and can commit one instruction per cycle. Draw the pipeline diagram of the following code sequence executing.
\begin{lstlisting}
MUL R6, R7, R8 
ADD R9, R10, R11 
ADD R11, R12, R13 
ADD R13, R14, R15 
ADD R19, R13, R10 
LW R2, R3
ADD R12, R16, R19 
LW R5, R2
ADD R15, R20, R21
\end{lstlisting}

% \solution Your solution...

\problem{20}{100} Assume architecture I2OI from lecture. Draw the state of the scoreboard when instruction 3 is in the I (Issue) stage of the pipeline.
\begin{lstlisting}
0: MUL R6, R7, R8
1: ADD R9, R6, R11
2: MUL R7, R1, R2
3: LW R10, R12
\end{lstlisting}

% \solution Your solution...


%  ASSIGNMENT AND SOLUTIONS GO HERE:

\section*{References}
\printbibliography[heading=none]

%  THIS ENDS THE EXAMPLES. DON'T DELETE THE FOLLOWING LINE:

\end{document}





